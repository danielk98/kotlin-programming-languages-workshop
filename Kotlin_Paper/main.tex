
\documentclass{article}
\usepackage[utf8]{inputenc}
\usepackage{amsmath}
\usepackage{amsfonts}
\usepackage{graphicx}
\usepackage{float}
\usepackage{a4wide}
\PassOptionsToPackage{hyphens}{url}\usepackage{hyperref}
\usepackage[ngerman]{babel} % Deutsch (neue Rechtschreibung)

\graphicspath{{Images/}}
\DeclareGraphicsExtensions{.pdf,.png,.jpg}

\title{Seminar Programmiersprachen: Kotlin}
\author{
  Daniel Karl\\
  {\small Universität Siegen}
}

\begin{document}
\maketitle

\begin{abstract}
 Kotlin ist eine von Jetbrains entwicklete Programmiersprache....
\end{abstract}

\section{Einleitung}
Wer, Wann, Warum entwicklet? Key-Features, Google unterstützt, Wie wird es heute weiterentwickelt. Motivation/Entstehung. Inspiriert von Java. Soll ein "einfacheres" Java sein. JetBrains\footnote{\url{https://www.jetbrains.com/de-de/} letzter Zugriff 26.11.2023}  

\section{Charakteristiken von Kotlin}
Design der Sprache, Type system, compilation, memory management, interpretation

\section{Besondere Konzepte}
evtl. mit ein paar kleinen Syntax Beispielen

\section{Verwendung von Kotlin}
Wieso wechselten Unternehmen wie Google, etc. zu Kotlin?
Wofür ist es geeignet/was kann die Sprache gut, wofür vielleicht eher nicht/ was kann die Sprache nicht so gut.

\subsection{Infrastruktur}
IDE, build, pacakgemanagement, docu

\subsection{Anwendungsbereiche}
Was kann man alles mit Kotlin machen (App-Entwicklung, Backend, Data Science, ML/Deep Learning etc) Wofür ist es besonders gut geeignet.
\subsection{Bekannte Projekte}
Android Apps (Netflix, Pinterest, Coursera, Altassian)\footnote{\url{https://developer.android.com/kotlin/stories/} letzter Zugriff 26.11.2023} , für APIs (Spring) davon nur ein oder zwei Beispiele
\section{Unterschiede zu Java}
Da Kotlin stark von Java inspiriert ist, macht es Sinn ein paar wesentliche Unterschiede hervorzuheben.

\section{Programm Implementierung}
Wie programmiert? Welche Bibliotheken wurden verwendet? Welche Konzepte waren besonderes hilfreich? Kommandozeilen Argumente parsen, Regex, Suchalgorithmus skizzieren, Besonderheiten während Implementierung
\subsection{Performance Optimierung}
Profiling Tools, mittels IntelliJ IDEA
\subsection{Performance Vergleich}
Vor und nach der Optimierung.

\section{Fazit}


\bibliographystyle{unsrt}
\bibliography{references}

\end{document}
