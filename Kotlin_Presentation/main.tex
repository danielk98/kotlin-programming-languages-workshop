\documentclass{beamer}
\usetheme[compress]{Singapore}
\setbeamertemplate{navigation symbols}{}%remove navigation symbols
%remove progression bar
\defbeamertemplate*{headline}{miniframes theme sections only}
{
    \begin{beamercolorbox}[colsep=1.5pt]{upper separation line head}
    \end{beamercolorbox}
}
\usepackage{cite}
\usepackage{setspace}
\usepackage[multiple]{footmisc}
\usepackage{algpseudocode}
\usepackage{xcolor}
\usepackage{enumerate}
\usepackage{graphicx}
\usepackage[english]{babel}
\usepackage[style=verbose,backend=bibtex]{biblatex}
%\addbibresource{references.bib}

\graphicspath{{images/}}
\DeclareGraphicsExtensions{.pdf,.png,.jpg}

\renewcommand{\footnotesize}{\tiny}

\addtobeamertemplate{navigation symbols}{}{%
    \usebeamerfont{footline}%
    \usebeamercolor[fg]{footline}%
    \hspace{1em}%
    \insertframenumber/\inserttotalframenumber
}

\title{Kotlin}
\subtitle{Seminar Programmiersprachen}
\author[D. Karl]
{Daniel Karl}
\date{5. Februar 2024}

\setbeamertemplate{frametitle}[default][left]
\setbeamertemplate{itemize items}[square]
\setbeamertemplate{itemize subitem}[square]
\setbeamertemplate{itemize subsubitem}[item]

\begin{document}
\maketitle


\begin{frame}
\frametitle{Geschichte}
\begin{itemize}
    \item Kotlin wurde von JetBrains entwickelt
    \item \textbf{Intention:} eine moderne Alternative zu Java
    \item 2010:
    \begin{itemize}
        %Beginn der Entwicklung von Kotlin
        \item Projekte wurden zuvor mit Java entwickelt
        \item wachsende Codebasis führte zu erschwerter Wartbarkeit
        \item Fokus auf Prägnanz und Interoperabilität mit Java
        %keine geeignete Alternative JVM-Sprache zu der Zeit
    \end{itemize}
    \item 2016:
    \begin{itemize}
        \item Veröffentlichung der ersten Version von Kotlin
    \end{itemize}
    \item 2017:
    \begin{itemize}
        \item Google berücksichtigt Kotlins Features bei der Weiterentwicklung der Android-Plattform %first-class support
    \end{itemize}
    \item Heute:
    \begin{itemize}
        \item wird aktiv weiterentwickelt
        \item aktuelle Version: 1.9.22
    \end{itemize}
\end{itemize}
\end{frame}


\begin{frame}
\frametitle{Typsystem}
\begin{itemize}
\onehalfspacing
    \item statisches Typsystem
    \item Typinferenz macht explizite Typangabe optional
    \item nullbare und nicht-nullbare Typen % ohne Typangabe, infereiert der Kompieler einen nicht-nullbaren Typen
    \item implizite Typkonversion nur auf Supertypen
    \item alle Typen in Kotlin basieren auf einer Klassendefinition
\end{itemize}
\end{frame}

\begin{frame}
\frametitle{Interoperabilität mit Java}
\begin{itemize}
\onehalfspacing
    \item Quellcode wird zu Java Bytecode übersetzt, wenn die JVM als Zielplattform gewählt wurde
    \item Java Klassen können direkt in Kotlin Projekte eingebunden werden
    \item nullbare Java Objekte werden zu Plattformtypen konvertiert
    \begin{itemize}
        \item besonderer Typ der nullbar und nicht-nullbar sein kann
        \item erleichtert Einbindung von Java Code
    \end{itemize}
    \item Unterstützung von sämtliche Java Bibliotheken
\end{itemize}
\end{frame}

\begin{frame}
\frametitle{Prägnanz}
\begin{itemize}
\onehalfspacing
    \item $val$ und $var$ zum Deklarieren von Variablen
    %\item explizite Typangaben sind optional
    \item getter und setter werden automatisch generiert
    \item \textbf{Datenklassen} bieten eine vereinfachte Schreibweise für Klassen die nur aus Properties bestehen
    \item \textbf{Higher-Order Funktionen}, nehmen Funktion als Parameter und/oder geben Funktion zurück. Erleichtern bspw. das Arbeiten mit Collections.
    \item \textbf{Extensions}, erweitern Klassen ohne Vererbung
\end{itemize}
\end{frame}


\begin{frame}
\frametitle{Coroutines}
\begin{itemize}
    \item Coroutines erlauben es effizient laufende asynchrone Programme zu schreiben
    %\item können als eine Art ressourcenschonender Thread gesehen werden
    \item werden innerhalb eines Threads ausgeführt und besitzen ihren eigenen Kontext
    \item lassen sich mit suspendierenden Funktionen anhalten
    \item in der Zwischenzeit können andere Coroutines im Thread ausgeführt werden
    \item sind unabhängig von einem Thread
\end{itemize}
\end{frame}


\begin{frame}
\frametitle{Infrastruktur}
\begin{itemize}
\onehalfspacing
    \item Buildsysteme: Gradle, Maven
    \item Dokumentation: KDoc (analog zu JavaDoc)
    \item Entwicklungsumgebung: IntelliJ, Android Studio, Eclipse
    \item Java to Kotlin converter (J2K), konvertiert automatisch Java Klassen zu Kotlin Dateien
    \item IntelliJ Profiler, als Werkzeug zur Optimierung von Speichernutzung und Laufzeit eines Programms
\end{itemize}
\end{frame}

\begin{frame}
\frametitle{Anwendungsbereiche}
\begin{itemize}
\onehalfspacingy
    \item Android App Entwicklung: von Google empfohlene Programmiersprache
    \item Desktop- und serverbasierte Anwendungen
    \item Multiplatform (Android und iOS)
    \item Data Science und Machine Learning
\end{itemize}
\end{frame}

\begin{frame}
\frametitle{Bekannte Projekte}
\begin{itemize}
\onehalfspacing
    \item Google Home App, Duolingo Android App
    \item Umstieg von Java zu Kotlin
    \item verbesserte Wartbarkeit und deutliche Reduzierung der Codezeilen
\end{itemize}
\end{frame}

\begin{frame}
\frametitle{Implementierung}
\begin{itemize}
\onehalfspacing
    \item \textbf{rekursive Dateisuche:}
    \begin{itemize}
        \item existierende Java Bibliotheken enthalten hilfreiche Kotlin-Extension Funktionen (isRegularFile(), isDirectory())
    \end{itemize}
    \item \textbf{Erkennen von Binärdateien:}
    \begin{itemize}
        \item Binärdateien habe typischerweise vermehrt Steuerzeichen und ASCII Zeichen aus dem Erweiterten ASCII Zeichen Satz
        \item Algorithmus nimmt 1024 Bytes einer Datei und zählt die typischen und untypische ASCII Zeichen
        \item falls über 50\% typische ASCII Zeichen, wird die Datei als Binärdatei betrachtet
    \end{itemize}
\end{itemize}
\end{frame}

\begin{frame}
\frametitle{Implementierung}
\onehalfspacing
\begin{itemize}
    \item \textbf{Suchen von regulären Ausdrücken:}
    \begin{itemize}
        \item Einbindung des java.io Pakets %Extension Funktion zum Einlesen der Datein verwendet
        \item vereinfachte Iteration durch die Textzeilen einer Datei mithilfe von Higher-Order Extension Funktionen
        \item Suche von regulären Ausdrücke mit dem java.util.regex Paket
    \end{itemize}

    \item \textbf{Command Line Interface:}
    \begin{itemize}
        \item mit der externen clikt Bibliothek umgesetzt
    \end{itemize}

\end{itemize}
\end{frame}




\begin{frame}
\frametitle{}
{\Large \centering Vielen Dank für Ihre Aufmerksamkeit!\par}
\end{frame}

\end{document}
